\documentclass[preprint]{sigplanconf}

%% ODER: format ==         = "\mathrel{==}"
%% ODER: format /=         = "\neq "
%
%
\makeatletter
\@ifundefined{lhs2tex.lhs2tex.sty.read}%
  {\@namedef{lhs2tex.lhs2tex.sty.read}{}%
   \newcommand\SkipToFmtEnd{}%
   \newcommand\EndFmtInput{}%
   \long\def\SkipToFmtEnd#1\EndFmtInput{}%
  }\SkipToFmtEnd

\newcommand\ReadOnlyOnce[1]{\@ifundefined{#1}{\@namedef{#1}{}}\SkipToFmtEnd}
\usepackage{amstext}
\usepackage{amssymb}
\usepackage{stmaryrd}
\DeclareFontFamily{OT1}{cmtex}{}
\DeclareFontShape{OT1}{cmtex}{m}{n}
  {<5><6><7><8>cmtex8
   <9>cmtex9
   <10><10.95><12><14.4><17.28><20.74><24.88>cmtex10}{}
\DeclareFontShape{OT1}{cmtex}{m}{it}
  {<-> ssub * cmtt/m/it}{}
\newcommand{\texfamily}{\fontfamily{cmtex}\selectfont}
\DeclareFontShape{OT1}{cmtt}{bx}{n}
  {<5><6><7><8>cmtt8
   <9>cmbtt9
   <10><10.95><12><14.4><17.28><20.74><24.88>cmbtt10}{}
\DeclareFontShape{OT1}{cmtex}{bx}{n}
  {<-> ssub * cmtt/bx/n}{}
\newcommand{\tex}[1]{\text{\texfamily#1}}	% NEU

\newcommand{\Sp}{\hskip.33334em\relax}


\newcommand{\Conid}[1]{\mathit{#1}}
\newcommand{\Varid}[1]{\mathit{#1}}
\newcommand{\anonymous}{\kern0.06em \vbox{\hrule\@width.5em}}
\newcommand{\plus}{\mathbin{+\!\!\!+}}
\newcommand{\bind}{\mathbin{>\!\!\!>\mkern-6.7mu=}}
\newcommand{\rbind}{\mathbin{=\mkern-6.7mu<\!\!\!<}}% suggested by Neil Mitchell
\newcommand{\sequ}{\mathbin{>\!\!\!>}}
\renewcommand{\leq}{\leqslant}
\renewcommand{\geq}{\geqslant}
\usepackage{polytable}

%mathindent has to be defined
\@ifundefined{mathindent}%
  {\newdimen\mathindent\mathindent\leftmargini}%
  {}%

\def\resethooks{%
  \global\let\SaveRestoreHook\empty
  \global\let\ColumnHook\empty}
\newcommand*{\savecolumns}[1][default]%
  {\g@addto@macro\SaveRestoreHook{\savecolumns[#1]}}
\newcommand*{\restorecolumns}[1][default]%
  {\g@addto@macro\SaveRestoreHook{\restorecolumns[#1]}}
\newcommand*{\aligncolumn}[2]%
  {\g@addto@macro\ColumnHook{\column{#1}{#2}}}

\resethooks

\newcommand{\onelinecommentchars}{\quad-{}- }
\newcommand{\commentbeginchars}{\enskip\{-}
\newcommand{\commentendchars}{-\}\enskip}

\newcommand{\visiblecomments}{%
  \let\onelinecomment=\onelinecommentchars
  \let\commentbegin=\commentbeginchars
  \let\commentend=\commentendchars}

\newcommand{\invisiblecomments}{%
  \let\onelinecomment=\empty
  \let\commentbegin=\empty
  \let\commentend=\empty}

\visiblecomments

\newlength{\blanklineskip}
\setlength{\blanklineskip}{0.66084ex}

\newcommand{\hsindent}[1]{\quad}% default is fixed indentation
\let\hspre\empty
\let\hspost\empty
\newcommand{\NB}{\textbf{NB}}
\newcommand{\Todo}[1]{$\langle$\textbf{To do:}~#1$\rangle$}

\EndFmtInput
\makeatother
%
%
%
%
%
%
% This package provides two environments suitable to take the place
% of hscode, called "plainhscode" and "arrayhscode". 
%
% The plain environment surrounds each code block by vertical space,
% and it uses \abovedisplayskip and \belowdisplayskip to get spacing
% similar to formulas. Note that if these dimensions are changed,
% the spacing around displayed math formulas changes as well.
% All code is indented using \leftskip.
%
% Changed 19.08.2004 to reflect changes in colorcode. Should work with
% CodeGroup.sty.
%
\ReadOnlyOnce{polycode.fmt}%
\makeatletter

\newcommand{\hsnewpar}[1]%
  {{\parskip=0pt\parindent=0pt\par\vskip #1\noindent}}

% can be used, for instance, to redefine the code size, by setting the
% command to \small or something alike
\newcommand{\hscodestyle}{}

% The command \sethscode can be used to switch the code formatting
% behaviour by mapping the hscode environment in the subst directive
% to a new LaTeX environment.

\newcommand{\sethscode}[1]%
  {\expandafter\let\expandafter\hscode\csname #1\endcsname
   \expandafter\let\expandafter\endhscode\csname end#1\endcsname}

% "compatibility" mode restores the non-polycode.fmt layout.

\newenvironment{compathscode}%
  {\par\noindent
   \advance\leftskip\mathindent
   \hscodestyle
   \let\\=\@normalcr
   \(\pboxed}%
  {\endpboxed\)%
   \par\noindent
   \ignorespacesafterend}

\newcommand{\compaths}{\sethscode{compathscode}}

% "plain" mode is the proposed default.
% It should now work with \centering.
% This required some changes. The old version
% is still available for reference as oldplainhscode.

\newenvironment{plainhscode}%
  {\hsnewpar\abovedisplayskip
   \advance\leftskip\mathindent
   \hscodestyle
   \let\hspre\(\let\hspost\)%
   \pboxed}%
  {\endpboxed%
   \hsnewpar\belowdisplayskip
   \ignorespacesafterend}

\newenvironment{oldplainhscode}%
  {\hsnewpar\abovedisplayskip
   \advance\leftskip\mathindent
   \hscodestyle
   \let\\=\@normalcr
   \(\pboxed}%
  {\endpboxed\)%
   \hsnewpar\belowdisplayskip
   \ignorespacesafterend}

% Here, we make plainhscode the default environment.

\newcommand{\plainhs}{\sethscode{plainhscode}}
\newcommand{\oldplainhs}{\sethscode{oldplainhscode}}
\plainhs

% The arrayhscode is like plain, but makes use of polytable's
% parray environment which disallows page breaks in code blocks.

\newenvironment{arrayhscode}%
  {\hsnewpar\abovedisplayskip
   \advance\leftskip\mathindent
   \hscodestyle
   \let\\=\@normalcr
   \(\parray}%
  {\endparray\)%
   \hsnewpar\belowdisplayskip
   \ignorespacesafterend}

\newcommand{\arrayhs}{\sethscode{arrayhscode}}

% The mathhscode environment also makes use of polytable's parray 
% environment. It is supposed to be used only inside math mode 
% (I used it to typeset the type rules in my thesis).

\newenvironment{mathhscode}%
  {\parray}{\endparray}

\newcommand{\mathhs}{\sethscode{mathhscode}}

% texths is similar to mathhs, but works in text mode.

\newenvironment{texthscode}%
  {\(\parray}{\endparray\)}

\newcommand{\texths}{\sethscode{texthscode}}

% The framed environment places code in a framed box.

\def\codeframewidth{\arrayrulewidth}
\RequirePackage{calc}

\newenvironment{framedhscode}%
  {\parskip=\abovedisplayskip\par\noindent
   \hscodestyle
   \arrayrulewidth=\codeframewidth
   \tabular{@{}|p{\linewidth-2\arraycolsep-2\arrayrulewidth-2pt}|@{}}%
   \hline\framedhslinecorrect\\{-1.5ex}%
   \let\endoflinesave=\\
   \let\\=\@normalcr
   \(\pboxed}%
  {\endpboxed\)%
   \framedhslinecorrect\endoflinesave{.5ex}\hline
   \endtabular
   \parskip=\belowdisplayskip\par\noindent
   \ignorespacesafterend}

\newcommand{\framedhslinecorrect}[2]%
  {#1[#2]}

\newcommand{\framedhs}{\sethscode{framedhscode}}

% The inlinehscode environment is an experimental environment
% that can be used to typeset displayed code inline.

\newenvironment{inlinehscode}%
  {\(\def\column##1##2{}%
   \let\>\undefined\let\<\undefined\let\\\undefined
   \newcommand\>[1][]{}\newcommand\<[1][]{}\newcommand\\[1][]{}%
   \def\fromto##1##2##3{##3}%
   \def\nextline{}}{\) }%

\newcommand{\inlinehs}{\sethscode{inlinehscode}}

% The joincode environment is a separate environment that
% can be used to surround and thereby connect multiple code
% blocks.

\newenvironment{joincode}%
  {\let\orighscode=\hscode
   \let\origendhscode=\endhscode
   \def\endhscode{\def\hscode{\endgroup\def\@currenvir{hscode}\\}\begingroup}
   %\let\SaveRestoreHook=\empty
   %\let\ColumnHook=\empty
   %\let\resethooks=\empty
   \orighscode\def\hscode{\endgroup\def\@currenvir{hscode}}}%
  {\origendhscode
   \global\let\hscode=\orighscode
   \global\let\endhscode=\origendhscode}%

\makeatother
\EndFmtInput
%


\usepackage{enumerate}
\usepackage{subfigure}
\usepackage{fancyvrb}
\usepackage{stmaryrd}
\usepackage{amsthm}
\usepackage{amsmath}
\usepackage{xypic}
\usepackage{multirow}
\usepackage{hyperref}
\usepackage{url}
\usepackage{color}

\ifdefined\nolhs
\DefineVerbatimEnvironment{code}{Verbatim}{fontsize=\small}
\else
\fi

\bibliographystyle{amsalpha}

\newcommand{\note}[1]{{\color{blue}{#1}}}

\newtheorem{lemma}{Lemma}
\newtheorem{theorem}{Theorem}
\theoremstyle{definition}
\newtheorem{definition}{Definition}

\newcommand{\play}{\mathsf{play}\;}
\newcommand{\playOp}{\textsf{play}}

\newcommand{\sleep}{\mathsf{sleep}\;}
\newcommand{\sleepOp}{\textsf{sleep}}

\newcommand{\ksleep}{\mathsf{kernelSleep}\;}
\newcommand{\ksleepOp}{\textsf{kernelSleep}}

\newcommand{\lang}{SonicPi}

\newcommand{\vtime}[1]{[#1]_{\mathsf{v}}}
\newcommand{\etime}[1]{[#1]_{\mathsf{t}}}

\newcommand{\interp}[1]{\llbracket{#1}\rrbracket}

\newcommand{\ie}{\emph{i.e.}}
\newcommand{\eg}{\emph{e.g.}}

\authorinfo{Sam Aaron}{}{}
\authorinfo{Dominic Orchard}{}{}
\title{A programming model for temporal coordination (in music)}

\begin{document}
\maketitle

\section{Introduction}
\label{sec:introduction}

\note{Introduction to SonicPi}

The underlying programming model of SonicPi provides a way to separate
the ordering of effects from the timing of
effects. Figure~\ref{three-chord-example} shows an example program
where three chords are played in sequence, combining simple notions of 
parallel, timed, and ordered effects. The first three statements play
the notes of a C major chord in parallel. A \sleepOp{} statement then 
provides a \emph{barrier}




Thus ``$\sleep{} t$'' communicates that, after it has been evaluated, at least 
$t$ seconds has elapsed since the last \sleepOp{}. This provides a minimum
time. In between calls to \sleepOp{}, any other statements can (with some limits)
be considered task parallel. 

In \lang{}, it is possible that a computation proceeding a \sleepOp{}
can overrun; that is, run longer than the sleep time.  Thus, the
programming model is not suitable for realtime systems requiring hard
deadlines but \sleepOp{} instead provides a \emph{soft deadline} (using
the terminology of Hansson and Jonsson~\cite{hansson1994logic}).


\begin{itemize}
\item A computational model of virtual and real time 
      structured by an \emph{applicative functor} abstraction.

\item We show that the applicative approach extends
      to a monadic approach, which can be composed with additional
      monads to capture other useful notions of effect in \lang{} programs,
      such as random numbers (Section~\ref{}).
\end{itemize}

Previously reported on the language~\cite{aaron2013sonic}

\begin{figure}[t]
\subfigure[Three chord program in \lang{}]{
\begin{minipage}{0.46\linewidth}
\[
\hspace{-6em}
\begin{array}{l}
\play C \\
\play E \\ 
\play G \\
\sleep 1 \\
\play F \\
\play A \\
\play C \\
\sleep 0.5 \\
\play G \\
\play B \\
\play D \\
\end{array}
\]
\end{minipage}
\label{three-chord-example}
}
\subfigure[Timing of the three chord program]{
\begin{minipage}{0.46\linewidth}
\note{insert nice diagram that shows when the notes
occur over the 1.5s duration} \\
\end{minipage}
\label{three-chord-timing}
}
\caption{Playing three chords (C major, F major, G major) 
in \lang{} with the second two chords played
closer together by $0.5s$.}
\end{figure}

\subsection{Examples}
\label{sec:examples}

\note{Show a few more example programs here that
demonstrate the programming model.}

Figure~\ref{sleep-examples} shows four similar programs which each
have different internal behaviours for \sleepOp, illustrating the
semantics of \sleepOp{}. The first three take 3s to execute and the
last takes 4s to execute, with the behaviours:
%
\begin{enumerate}[(a)]
\item{3s -- sleeps for 1s then sleeps for 2s (two sleeps performed);}
\item{3s -- performs a computation lasting 1s, ignores
the first \sleepOp{} since its minimum deadline has been reached, 
and then sleeps for 2s (one sleep performed);}
\item{3s -- performs a computation lasting 2s, which means that
the first \sleepOp{} is ignored, and the second \sleepOp{} waits
for only 1s to reach its minimum deadline (half a sleep performed);}
\item{4s -- performs a computation lasting 2s, thus 
the first \sleepOp{} is ignored, then performs a computation lasting
2s, thus the second \sleepOp{} is ignored (no sleeps performed).}
\end{enumerate}


\begin{figure}[t]
\subfigure[Two sleeps]{
\begin{minipage}{0.18\linewidth}
\[
\hspace{-1em}
\begin{array}{l}
\sleep 1 \\
\sleep 2 \\ \\ \\ \\
\end{array}
\]
\end{minipage}
\label{sleep-examples:a}
}
\rule[-2em]{0.3pt}{5em}
%\hspace{1em}
% takes 3
\subfigure[One sleep]{
\begin{minipage}{0.23\linewidth}
\begin{center}
\[
\hspace{-0.5em}
\begin{array}{l}
\ldots \; \textit{\# lasts 1s} \\
\sleep 1 \\
\sleep 2 \\ \\  \\
\end{array}
\]
\end{center}
\end{minipage}
\label{sleep-examples:b}
}
\rule[-2em]{0.3pt}{5em}
%\hspace{1em}
% takes 3s
\subfigure[Half a sleep]{
\begin{minipage}{0.23\linewidth}
\begin{center}
\[
\hspace{-0.5em}
\begin{array}{l}
\ldots \; \textit{\# lasts 2s} \\
\sleep 1 \\
\sleep 2 \\ \\ \\
\end{array}
\]
\end{center}
\end{minipage}
\label{sleep-examples:c}
% takes 6
}
\rule[-2em]{0.3pt}{5em}
\subfigure[No sleeps]{
\begin{minipage}{0.23\linewidth}
\begin{center}
\[
\hspace{-0.5em}
\begin{array}{l}
\ldots \; \textit{\# lasts 2s} \\
\sleep 1 \\
\ldots \; \textit{\# lasts 2s} \\
\sleep 2 \\  \\
\end{array}
\]
\end{center}
\end{minipage}
}

\caption{Example programs with different \sleepOp{} behaviours}
\label{sleep-examples}
\end{figure}

\section{Model}

\paragraph{Terminology and notation}
We refer to closed sequences of statements (\ie{}, without free
variables) as \emph{programs}. Throughout, $P$, $Q$ range over programs,
and $s, t$ range over times (usually in seconds).

\subsection{Virtual time and real time}

The programming model of \lang{} distinguishes between the
\emph{actual time} elapsed since the start of a program $P$, which we
write as $\etime{P}$, and the \emph{virtual time} which is advanced by
\sleepOp{} statements, written $\vtime{P}$.

Since \sleepOp{} is the only operation that changes the virtual
time, we can specification for $\vtime{-}$ over all programs:
%
\begin{definition}
Virtual time is specified for statements of \lang{} programs 
by the following (ordered) cases:
%
\begin{align*}
\vtime{\sleep t} & = t \\ 
\vtime{P; Q} & = \vtime{P} + \vtime{Q} \\
\vtime{-} & = 0
\end{align*}
%
\ie{}, the virtual time is $0$ 
for any statment other than \sleepOp{} or sequential composition.
\label{sleep-spec}
\end{definition}

Equality of computation times is difficult. In order to not sweep this
issue entirely under the carpet, we will use the relation $\approx$,
where $s \approx t$ means that $s$ and $t$ are either exactly equal or
differ by some amount $\epsilon$.\note{Discuss this further, may be
  able to say later that in some cases $\epsilon$ is the scheduling
  time for play statments?}  The virtual time and actual time of a
single sleep statement (in isolation) are roughly the same, \ie{},
$\vtime{\sleep t} \approx \etime{\sleep t}$ and thus $\vtime{\sleep t}
\approx t$ by the specification in Definition~\ref{sleep-spec}.

We should be clear however, that this approximate equality does not always
hold. We can only say that it holds when \sleepOp{} is used in isolation, that is, 
when it is the only statement in a program. As shown by the examples
of Section~\ref{sec:examples}, the use of $\sleep t$ in a program 
does not mean that a program necessarily waits for $t$ seconds-- 
depending on the context, it may wait for anywhere between $0$ and $t$ seconds.
 
For convenience, and to contrast with \sleepOp{}, we'll use an additional
statement \ksleepOp{} here (which is not available in the actual language)
 which always sleeps for the number of seconds specified by its parameter.


\begin{lemma}
For some program $P$ and time $t$: 
%%
\begin{align*}
\etime{P; \sleep{} t} = 
 \begin{cases}
   \etime{P} & (\vtime{P} + t) < \etime{P} \\
   \vtime{P} + t  & \textit{otherwise}
 \end{cases}
\end{align*}
%\begin{align*}
%\etime{P; \sleep{} t} = 
% \begin{cases}
%   \etime{P} & t < \etime{P} \\
%   \vtime{P} + t & t \geq \etime{P}
% \end{cases}
%\end{align*}
\label{lem:sleep-R}
\end{lemma}

\begin{lemma}
For some program $P$ and time $t$:
%%
\begin{align*}
\etime{\sleep{} t; P} = t + \etime{P}
\end{align*}
\label{lem:sleep-L}
\end{lemma}

\begin{lemma}
For all programs $P$ then $\etime{P} \geq \vtime{P}$. 
\label{lemma1}
\end{lemma}

\begin{theorem}
For all programs $P$ and $Q$ then:
%%
\begin{equation}
\vtime{P} + \vtime{Q} \leq \etime{P; Q} \leq \etime{P} + \etime{Q}
\end{equation}
\label{theorem:main}
\end{theorem}


From these lemmas we can reason about the evaluation time of programs.
For example, consider subprograms $A$, $B$, $C$ where 
$\etime{A} = t_1$, $\etime{B} = t_2$, $\etime{C} = t_3$ and
$\vtime{A} = \vtime{B} \vtime{C} = 0$, interposed with two
sleep statements of duration $s_1$ and $s_2$:
%
\begin{equation}
\begin{array}{l}
A \\
\sleep s_1 \\
B  \\
\sleep s_2 \\
C
\end{array}
\label{example:time1}
\end{equation}
%%
Then by the above lemmas, we see that $\etime{eq. \eqref{example:time1}} = 
s_1 + s_2 + t_3$, iff $t_1 \leq 1$ and $t_2 \leq 2$.

%\begin{equation*}
%\begin{array}{lllll}
%A & \multirow{2}{*}{\rule[1em]{0.6pt}{1.2em}} & \multirow{2}{*}{$t_1$} & 
%\multirow{4}{*}{\rule[1em]{0.6pt}{4em}} & \multirow{4}{*}{$t_1 + t_2 + 3$}
%\\
%\emph{sleep} \; 1 \qquad \\
%B &  \multirow{2}{*}{\rule[1em]{0.6pt}{1.2em}} & \multirow{2}{*}{$t_2$} \\
%\emph{sleep} \; 2
%\end{array}
%\end{equation*}


\newcommand{\TM}{\mathsf{TM}}

\subsection{Monadic structure on computation}

In the following, we use Haskell as our meta language for the
semantics (since it provides convenient syntax for working with
monads)\footnote{The source code for the model is avilable at
  \url{https://github.com/dorchard/time-monad}}.
\lang{} computations are modelled by the \emph{Temporal} data type, defined:
%%
\begin{hscode}\SaveRestoreHook
\column{B}{@{}>{\hspre}l<{\hspost}@{}}%
\column{5}{@{}>{\hspre}l<{\hspost}@{}}%
\column{E}{@{}>{\hspre}l<{\hspost}@{}}%
\>[B]{}\mathbf{data}\;\Conid{Temporal}\;\Varid{a}\mathrel{=}{}\<[E]%
\\
\>[B]{}\hsindent{5}{}\<[5]%
\>[5]{}\Conid{T}\;((\Conid{Time},\Conid{Time})\to (\Conid{VTime}\to \Conid{IO}\;(\Varid{a},\Conid{VTime}))){}\<[E]%
\ColumnHook
\end{hscode}\resethooks
%
Thus, temporal computations map a pair of two times, which will be
the start time of the computation and current time, to a stateful
computation with a single location storing a virtual time, over the
\emph{IO} type.  The \emph{IO} computation provides underlying access
to the actual time from kernel.

The \emph{Monad} instance for \emph{Temporal} is then as follows:
%
\begin{hscode}\SaveRestoreHook
\column{B}{@{}>{\hspre}l<{\hspost}@{}}%
\column{3}{@{}>{\hspre}l<{\hspost}@{}}%
\column{16}{@{}>{\hspre}l<{\hspost}@{}}%
\column{21}{@{}>{\hspre}l<{\hspost}@{}}%
\column{25}{@{}>{\hspre}l<{\hspost}@{}}%
\column{37}{@{}>{\hspre}l<{\hspost}@{}}%
\column{E}{@{}>{\hspre}l<{\hspost}@{}}%
\>[B]{}\mathbf{instance}\;\Conid{Monad}\;\Conid{Temporal}\;\mathbf{where}{}\<[E]%
\\
\>[B]{}\hsindent{3}{}\<[3]%
\>[3]{}\Varid{return}\;\Varid{a}{}\<[16]%
\>[16]{}\mathrel{=}\Conid{T}\;(\lambda \anonymous \to \lambda \Varid{vT}\to \Varid{return}\;(\Varid{a},\Varid{vT})){}\<[E]%
\\[\blanklineskip]%
\>[B]{}\hsindent{3}{}\<[3]%
\>[3]{}\\[-1.5em]{}\<[E]%
\\
\>[B]{}\hsindent{3}{}\<[3]%
\>[3]{}(\Conid{T}\;\Varid{p})\bind \Varid{q}{}\<[16]%
\>[16]{}\mathrel{=}\Conid{T}\;(\lambda (\Varid{startT},\Varid{nowT})\to \lambda \Varid{vT}\to {}\<[E]%
\\
\>[16]{}\hsindent{5}{}\<[21]%
\>[21]{}\mathbf{do}\;{}\<[25]%
\>[25]{}(\Varid{x},\Varid{vT'}){}\<[37]%
\>[37]{}\leftarrow \Varid{p}\;(\Varid{startT},\Varid{nowT})\;\Varid{vT}{}\<[E]%
\\
\>[25]{}\mathbf{let}\;(\Conid{T}\;\Varid{q'}){}\<[37]%
\>[37]{}\mathrel{=}\Varid{q}\;\Varid{x}{}\<[E]%
\\
\>[25]{}\Varid{thenT}{}\<[37]%
\>[37]{}\leftarrow \Varid{getCurrentTime}{}\<[E]%
\\
\>[25]{}\Varid{q'}\;(\Varid{startT},\Varid{thenT})\;\Varid{vT'}){}\<[E]%
\ColumnHook
\end{hscode}\resethooks
%
\begin{itemize}

\item \ensuremath{\Varid{return}\mathbin{::}\Varid{a}\to \Conid{Temporal}\;\Varid{a}} thus ignores the time parameters and
has the usual pure state behaviour: passing the state unchanged.
For the bind

\item \ensuremath{(\bind )\mathbin{::}\Conid{Temporal}\;\Varid{a}\to (\Varid{a}\to \Conid{Temporal}\;\Varid{b})\to \Conid{Temporal}\;\Varid{b}} thus
composes two computations together. 
The result of composing two temporal computations, with
start time \emph{starT}, current time \emph{nowT}, and virtual
time \emph{vT}, is the result of evaluating first the right 
hand side, and then the left hand side 


\end{itemize}

The following \ensuremath{\Varid{runTime}} operation executes a temporal computation
inside of the \emph{IO} monad by providing the start time of the
computation, and with virtual time 0:
%%
\begin{hscode}\SaveRestoreHook
\column{B}{@{}>{\hspre}l<{\hspost}@{}}%
\column{21}{@{}>{\hspre}l<{\hspost}@{}}%
\column{E}{@{}>{\hspre}l<{\hspost}@{}}%
\>[B]{}\Varid{runTime}\mathbin{::}\Conid{Temporal}\;\Varid{a}\to \Conid{IO}\;\Varid{a}{}\<[E]%
\\
\>[B]{}\Varid{runTime}\;(\Conid{T}\;\Varid{c})\mathrel{=}\mathbf{do}\;{}\<[21]%
\>[21]{}\Varid{startT}\leftarrow \Varid{getCurrentTime}{}\<[E]%
\\
\>[21]{}(\Varid{x},\anonymous )\leftarrow \Varid{c}\;(\Varid{startT},\Varid{startT})\;\mathrm{0}{}\<[E]%
\\
\>[21]{}\Varid{return}\;\Varid{x}{}\<[E]%
\ColumnHook
\end{hscode}\resethooks
%%
To illustrate the evalution of temporal computation and the
ordering and interleaving of calls to the operation system for the
current time, we consider a program:
%%
\begin{hscode}\SaveRestoreHook
\column{B}{@{}>{\hspre}l<{\hspost}@{}}%
\column{E}{@{}>{\hspre}l<{\hspost}@{}}%
\>[B]{}\Varid{runTime}\;(\mathbf{do}\;\{\mskip1.5mu \Varid{f};\Varid{g};\Varid{h};\mskip1.5mu\}){}\<[E]%
\ColumnHook
\end{hscode}\resethooks
(where \ensuremath{\Varid{f}\mathrel{=}\Conid{T}\;\Varid{f'},\Varid{g}\mathrel{=}\Conid{T}\;\Varid{g'},\Varid{h}\mathrel{=}\Conid{T}\;\Varid{h'}}) which 
desugars to the following, after some simplification: 
%%
\begin{hscode}\SaveRestoreHook
\column{B}{@{}>{\hspre}l<{\hspost}@{}}%
\column{5}{@{}>{\hspre}l<{\hspost}@{}}%
\column{15}{@{}>{\hspre}l<{\hspost}@{}}%
\column{E}{@{}>{\hspre}l<{\hspost}@{}}%
\>[B]{}\mathbf{do}\;{}\<[5]%
\>[5]{}\Varid{startT}{}\<[15]%
\>[15]{}\leftarrow \Varid{getCurrentTime}{}\<[E]%
\\
\>[5]{}(\anonymous ,\Varid{vT'}){}\<[15]%
\>[15]{}\leftarrow \Varid{f'}\;(\Varid{startT},\Varid{startT})\;\mathrm{0}{}\<[E]%
\\
\>[5]{}\Varid{thenT}{}\<[15]%
\>[15]{}\leftarrow \Varid{getCurrentTime}{}\<[E]%
\\
\>[5]{}(\anonymous ,\Varid{vT''})\leftarrow \Varid{g'}\;(\Varid{startT},\Varid{thenT})\;\Varid{vT'}{}\<[E]%
\\
\>[5]{}\Varid{thenT'}{}\<[15]%
\>[15]{}\leftarrow \Varid{getCurrentTime}{}\<[E]%
\\
\>[5]{}(\Varid{y},\anonymous ){}\<[15]%
\>[15]{}\leftarrow \Varid{h'}\;(\Varid{startT},\Varid{thenT'})\;\Varid{vT''})\;{}\<[E]%
\\
\>[5]{}\Varid{return}\;\Varid{y}{}\<[E]%
\ColumnHook
\end{hscode}\resethooks
%
The above illustrates the repeated calls to \ensuremath{\Varid{getCurrentTime}}, the
constant start time parameter, and the threading of virtual time state
throughout the computation. 

\begin{figure}[t]
\begin{hscode}\SaveRestoreHook
\column{B}{@{}>{\hspre}l<{\hspost}@{}}%
\column{7}{@{}>{\hspre}l<{\hspost}@{}}%
\column{18}{@{}>{\hspre}l<{\hspost}@{}}%
\column{24}{@{}>{\hspre}l<{\hspost}@{}}%
\column{28}{@{}>{\hspre}l<{\hspost}@{}}%
\column{E}{@{}>{\hspre}l<{\hspost}@{}}%
\>[B]{}\Varid{time}\mathbin{::}\Conid{Temporal}\;\Conid{Time}{}\<[E]%
\\
\>[B]{}\Varid{time}{}\<[7]%
\>[7]{}\mathrel{=}\Conid{T}\;(\lambda (\anonymous ,\Varid{nowT})\to \lambda \Varid{vT}\to \Varid{return}\;(\Varid{nowT},\Varid{vT})){}\<[E]%
\\[\blanklineskip]%
\>[B]{}\Varid{start}\mathbin{::}\Conid{Temporal}\;\Conid{Time}{}\<[E]%
\\
\>[B]{}\Varid{start}\mathrel{=}\Conid{T}\;(\lambda (\Varid{startT},\anonymous )\to \lambda \Varid{vT}\to \Varid{return}\;(\Varid{startT},\Varid{vT})){}\<[E]%
\\[\blanklineskip]%
\>[B]{}\Varid{getVirtualTime}\mathbin{::}\Conid{Temporal}\;\Conid{VTime}{}\<[E]%
\\
\>[B]{}\Varid{getVirtualTime}\mathrel{=}\Conid{T}\;(\lambda (\anonymous ,\anonymous )\to \lambda \Varid{vT}\to \Varid{return}\;(\Varid{vT},\Varid{vT})){}\<[E]%
\\[\blanklineskip]%
\>[B]{}\Varid{setVirtualTime}\mathbin{::}\Conid{VTime}\to \Conid{Temporal}\;(){}\<[E]%
\\
\>[B]{}\Varid{setVirtualTime}\;\Varid{vT}\mathrel{=}\Conid{T}\;(\lambda \anonymous \to \lambda \anonymous \to \Varid{return}\;((),\Varid{vT})){}\<[E]%
\\[\blanklineskip]%
\>[B]{}\Varid{kernelSleep}\mathbin{::}\Conid{RealFrac}\;\Varid{a}\Rightarrow \Varid{a}\to \Conid{Temporal}\;(){}\<[E]%
\\
\>[B]{}\Varid{kernelSleep}\;\Varid{t}\mathrel{=}{}\<[18]%
\>[18]{}\Conid{T}\;(\lambda (\anonymous ,\anonymous )\to \lambda \Varid{vT}\to {}\<[E]%
\\
\>[18]{}\hsindent{6}{}\<[24]%
\>[24]{}\mathbf{do}\;{}\<[28]%
\>[28]{}\Varid{threadDelay}\;(\Varid{round}\;(\Varid{t}\mathbin{*}\mathrm{1000000})){}\<[E]%
\\
\>[28]{}\Varid{return}\;((),\Varid{vT})){}\<[E]%
\ColumnHook
\end{hscode}\resethooks
\caption{Simple \emph{Temporal} computations, used  by the model}
\end{figure}

\paragraph{Interpreting \lang{} statements}

The following interpretation function $\interp{-}$ on \lang{} 
programs shows the mapping to the operations of the \emph{Temporal}
monad:
%%
\begin{align*}
\interp{\emph{statement}} & : \emph{Temporal} \, () \\
\interp{x = P; Q} & = \interp{P} \ensuremath{\bind (\lambda \Varid{x}\to } \interp{Q}) \\
\interp{P; Q} & = \interp{P} \ensuremath{\bind (\lambda \anonymous \to } \interp{Q}) \\
\interp{\sleep e} & = \emph{sleep} \, \interp{e}
\end{align*}
%%
Note that $\interp{-}$ is overloaded in the rule for \sleepOp{} for (pure) expressions. 
The concrete interpreation of other statements in the language, such as \playOp, is
elided here since it does not relate directly to the temporal semantics. 

The key primitive \emph{sleep} provides the semantics for \sleepOp{} as:
%%
\begin{hscode}\SaveRestoreHook
\column{B}{@{}>{\hspre}l<{\hspost}@{}}%
\column{20}{@{}>{\hspre}l<{\hspost}@{}}%
\column{22}{@{}>{\hspre}l<{\hspost}@{}}%
\column{28}{@{}>{\hspre}l<{\hspost}@{}}%
\column{30}{@{}>{\hspre}l<{\hspost}@{}}%
\column{E}{@{}>{\hspre}l<{\hspost}@{}}%
\>[B]{}\Varid{sleep}\mathbin{::}\Conid{VTime}\to \Conid{Temporal}\;(){}\<[E]%
\\
\>[B]{}\Varid{sleep}\;\Varid{delayT}\mathrel{=}\mathbf{do}\;{}\<[20]%
\>[20]{}\Varid{vT}{}\<[28]%
\>[28]{}\leftarrow \Varid{getVirtualTime}{}\<[E]%
\\
\>[20]{}\Varid{startT}{}\<[28]%
\>[28]{}\leftarrow \Varid{start}{}\<[E]%
\\
\>[20]{}\Varid{nowT}{}\<[28]%
\>[28]{}\leftarrow \Varid{time}{}\<[E]%
\\
\>[20]{}\mathbf{let}\;\Varid{vT'}{}\<[30]%
\>[30]{}\mathrel{=}\Varid{vT}\mathbin{+}\Varid{delayT}{}\<[E]%
\\
\>[20]{}\mathbf{let}\;\Varid{diffT}\mathrel{=}\Varid{diffTime}\;\Varid{nowT}\;\Varid{startT}{}\<[E]%
\\
\>[20]{}\mathbf{if}\;(\Varid{vT'}\mathbin{<}\Varid{diffT}){}\<[E]%
\\
\>[20]{}\hsindent{2}{}\<[22]%
\>[22]{}\mathbf{then}\;\Varid{return}\;(){}\<[E]%
\\
\>[20]{}\hsindent{2}{}\<[22]%
\>[22]{}\mathbf{else}\;\Varid{kernelSleep}\;(\Varid{vT'}\mathbin{-}\Varid{diffT}){}\<[E]%
\\
\>[20]{}\Varid{setVirtualTime}\;\Varid{vT'}{}\<[E]%
\ColumnHook
\end{hscode}\resethooks
%%
Thus, \emph{sleep} proceeds by calculating the elapsed time, \emph{diffT}, 
and the new virtual time \emph{vT'} (after the sleep). 
If the new virtual time is less than the elapsed time then no actual (kernel) 
sleeping happens. However, if the new virtual time is ahead of the elapsed time, 
then the process sleeps for the difference. Finally, the virtual time state is updated
with the new virtual time. 

\subsection{Soundness}

\paragraph{Lemma~\ref{lem:sleep-R}} 

\begin{hscode}\SaveRestoreHook
\column{B}{@{}>{\hspre}l<{\hspost}@{}}%
\column{4}{@{}>{\hspre}l<{\hspost}@{}}%
\column{7}{@{}>{\hspre}l<{\hspost}@{}}%
\column{15}{@{}>{\hspre}l<{\hspost}@{}}%
\column{16}{@{}>{\hspre}l<{\hspost}@{}}%
\column{E}{@{}>{\hspre}l<{\hspost}@{}}%
\>[B]{}\mathbf{do}\;\Varid{startT}\leftarrow \Varid{getCurrentTime}{}\<[E]%
\\
\>[B]{}\hsindent{4}{}\<[4]%
\>[4]{}(\Varid{x},\Varid{vT'}){}\<[16]%
\>[16]{}\leftarrow \Varid{p}\;(\Varid{startT},\Varid{startT})\;\mathrm{0}{}\<[E]%
\\
\>[B]{}\hsindent{4}{}\<[4]%
\>[4]{}\Varid{thenT}{}\<[16]%
\>[16]{}\leftarrow \Varid{getCurrentTime}{}\<[E]%
\\
\>[B]{}\hsindent{4}{}\<[4]%
\>[4]{}\mathbf{let}\;\Varid{vT''}{}\<[15]%
\>[15]{}\mathrel{=}\Varid{vT'}\mathbin{+}\Varid{t}{}\<[E]%
\\
\>[B]{}\hsindent{4}{}\<[4]%
\>[4]{}\mathbf{let}\;\Varid{diffT}\mathrel{=}\Varid{diffTime}\;\Varid{thenT}\;\Varid{startT}{}\<[E]%
\\
\>[B]{}\hsindent{4}{}\<[4]%
\>[4]{}\mathbf{if}\;(\Varid{vT''}\mathbin{<}\Varid{diffT}){}\<[E]%
\\
\>[4]{}\hsindent{3}{}\<[7]%
\>[7]{}\mathbf{then}\;\Varid{return}\;(){}\<[E]%
\\
\>[4]{}\hsindent{3}{}\<[7]%
\>[7]{}\mathbf{else}\;\Varid{kernelSleep}\;(\Varid{vT''}\mathbin{-}\Varid{diffT}){}\<[E]%
\\
\>[B]{}\hsindent{4}{}\<[4]%
\>[4]{}\Varid{setVirtualTime}\;\Varid{vT''}{}\<[E]%
\ColumnHook
\end{hscode}\resethooks

\begin{align*}
\etime{P; \sleep{} t} = 
 \begin{cases}
   \etime{P} & (\vtime{P} + t) < \etime{P}  \\
   \etime{P} + (\vtime{P} + t) - \etime{P}  & \textit{otherwise}
 \end{cases}
\end{align*}

\begin{align*}
\etime{P; \sleep{} t} = 
 \begin{cases}
   \etime{P} & (\vtime{P} + t) < \etime{P} \\
   \vtime{P} + t  &  \textit{otherwise}
 \end{cases}
\end{align*}


\paragraph{Lemma~\ref{lem:sleep-L}}
 $\etime{\sleep{} t; P} = t + \etime{P}$.
This lemma holds in the model, where \ensuremath{\Varid{runTime}\;(\mathbf{do}\;\{\mskip1.5mu \Varid{sleep}\;\Varid{t};\Conid{P}\mskip1.5mu\})} desugars and
simplifies to:
%
\begin{hscode}\SaveRestoreHook
\column{B}{@{}>{\hspre}l<{\hspost}@{}}%
\column{5}{@{}>{\hspre}l<{\hspost}@{}}%
\column{12}{@{}>{\hspre}l<{\hspost}@{}}%
\column{17}{@{}>{\hspre}l<{\hspost}@{}}%
\column{E}{@{}>{\hspre}l<{\hspost}@{}}%
\>[B]{}\mathbf{do}\;{}\<[5]%
\>[5]{}\Varid{startT}\leftarrow \Varid{getCurrentTime}{}\<[E]%
\\
\>[5]{}\Varid{kernelSleep}\;\Varid{t}{}\<[E]%
\\
\>[5]{}\mathbf{let}\;(\Conid{T}\;\Varid{p'}){}\<[17]%
\>[17]{}\mathrel{=}\Varid{p}\;(){}\<[E]%
\\
\>[5]{}\Varid{thenT}{}\<[12]%
\>[12]{}\leftarrow \Varid{getCurrentTime}{}\<[E]%
\\
\>[5]{}\Varid{p'}\;(\Varid{startT},\Varid{thenT})\;\Varid{t}{}\<[E]%
\ColumnHook
\end{hscode}\resethooks
%
Thus, we see that we \ensuremath{\Varid{kernelSleep}} for $t$, and then continue with $P$ at virtual
time $t$.


\paragraph{Quotienting by non-time dependent functions}

\note{This section is more of a marker for myself (Dom), I need
to think about this later, but basically it is about showing
that the monad laws hold for our semantics (but not in general)}

In L, there is no expression which returns the current time; 
 \emph{getTime} belongs only to the model, not to the language.
That is, for all expressions $e$, then the denotation 
$\interp{e}$ factors through 

\subsection{Subsets of the semantics}

For the examples of Section~\ref{sec:introduction}, the full structure
of monad is not needed to give their semantics as there is no using of
binding between statements (and thus no dataflow). In these case just
an \emph{applicative functor}~\cite{mcbride2008functional} or even a
monoid would suffice. These can be derived from the monad structure
on \emph{Temporal} since all monads are applicative functors and all 
monads define monoid over $m ()$.

\paragraph{Applicative subset}

Applicative functors are described by the following interface in
Haskell:
%%
\begin{hscode}\SaveRestoreHook
\column{B}{@{}>{\hspre}l<{\hspost}@{}}%
\column{4}{@{}>{\hspre}l<{\hspost}@{}}%
\column{10}{@{}>{\hspre}l<{\hspost}@{}}%
\column{E}{@{}>{\hspre}l<{\hspost}@{}}%
\>[B]{}\mathbf{class}\;\Conid{Functor}\;\Varid{f}\Rightarrow \Conid{Applicative}\;\Varid{f}\;\mathbf{where}{}\<[E]%
\\
\>[B]{}\hsindent{4}{}\<[4]%
\>[4]{}\Varid{pure}{}\<[10]%
\>[10]{}\mathbin{::}\Varid{a}\to \Varid{f}\;\Varid{a}{}\<[E]%
\\
\>[B]{}\hsindent{4}{}\<[4]%
\>[4]{}(<\!\!\!*\!\!\!>)\mathbin{::}\Varid{f}\;(\Varid{a}\to \Varid{b})\to \Varid{f}\;\Varid{a}\to \Varid{f}\;\Varid{b}{}\<[E]%
\ColumnHook
\end{hscode}\resethooks
%
The \emph{Applicative} class describes how to compose effectul
computations encoded as values of type $f\ a$ (the effectful
computation of a pure value of type $a$). Thus, \emph{pure} constructs
a trivially effectful computation from a pure value and \ensuremath{<\!\!\!*\!\!\!>} (akin to
application) takes an effectful computation of a function and an
effectful computation of an argument and evaluates the two effects.

Our \emph{Temporal} denotations have the applicative functor
structure with the following derivation in terms of the monad:
%
\begin{hscode}\SaveRestoreHook
\column{B}{@{}>{\hspre}l<{\hspost}@{}}%
\column{5}{@{}>{\hspre}l<{\hspost}@{}}%
\column{21}{@{}>{\hspre}l<{\hspost}@{}}%
\column{E}{@{}>{\hspre}l<{\hspost}@{}}%
\>[B]{}\mathbf{instance}\;\Conid{Functor}\;\Conid{Temporal}\;\mathbf{where}{}\<[E]%
\\
\>[B]{}\hsindent{5}{}\<[5]%
\>[5]{}\Varid{fmap}\;\Varid{f}\;\Varid{x}\mathrel{=}\mathbf{do}\;\{\mskip1.5mu \Varid{x'}\leftarrow \Varid{x};\Varid{return}\;(\Varid{f}\;\Varid{x'})\mskip1.5mu\}{}\<[E]%
\\[\blanklineskip]%
\>[B]{}\mathbf{instance}\;\Conid{Applicative}\;\Conid{Temporal}\;\mathbf{where}{}\<[E]%
\\
\>[B]{}\hsindent{5}{}\<[5]%
\>[5]{}\Varid{pure}\;\Varid{a}{}\<[21]%
\>[21]{}\mathrel{=}\Varid{return}\;\Varid{a}{}\<[E]%
\\
\>[B]{}\hsindent{5}{}\<[5]%
\>[5]{}\Varid{f}<\!\!\!*\!\!\!>\Varid{x}{}\<[21]%
\>[21]{}\mathrel{=}\mathbf{do}\;\{\mskip1.5mu \Varid{f'}\leftarrow \Varid{f};\Varid{x'}\leftarrow \Varid{x};\Varid{return}\;(\Varid{f'}\;\Varid{x'})\mskip1.5mu\}{}\<[E]%
\ColumnHook
\end{hscode}\resethooks
%
Note that the definition of \ensuremath{<\!\!\!*\!\!\!>} here orders the effects left-to-right.

The interpretation of sequential composition for statements (with no
dataflow) is then $\interp{P; Q} = (\lambda() \rightarrow
\interp{P}) <\!\!\!*\!\!\!> \interp{Q}$.

\paragraph{Monoid subset}

Alternatively, the full structure of an applicative functor is
not even needed in this restricted case. Instead, a monoid
over \emph{Temporal ()} would suffice:
%
\begin{hscode}\SaveRestoreHook
\column{B}{@{}>{\hspre}l<{\hspost}@{}}%
\column{4}{@{}>{\hspre}l<{\hspost}@{}}%
\column{5}{@{}>{\hspre}l<{\hspost}@{}}%
\column{19}{@{}>{\hspre}l<{\hspost}@{}}%
\column{E}{@{}>{\hspre}l<{\hspost}@{}}%
\>[B]{}\mathbf{class}\;\Conid{Monoid}\;\Varid{m}\;\mathbf{where}{}\<[E]%
\\
\>[B]{}\hsindent{4}{}\<[4]%
\>[4]{}\Varid{mempty}\mathbin{::}\Varid{m}{}\<[E]%
\\
\>[B]{}\hsindent{4}{}\<[4]%
\>[4]{}\Varid{mappend}\mathbin{::}\Varid{m}\to \Varid{m}\to \Varid{m}{}\<[E]%
\\[\blanklineskip]%
\>[B]{}\mathbf{instance}\;\Conid{Monoid}\;(\Conid{Temporal}\;())\;\mathbf{where}{}\<[E]%
\\
\>[B]{}\hsindent{5}{}\<[5]%
\>[5]{}\Varid{mempty}{}\<[19]%
\>[19]{}\mathrel{=}\Varid{return}\;(){}\<[E]%
\\
\>[B]{}\hsindent{5}{}\<[5]%
\>[5]{}\Varid{a}\mathbin{`\Varid{mappend}`}\Varid{b}\mathrel{=}\mathbf{do}\;\{\mskip1.5mu \Varid{a};\Varid{b};\Varid{return}\;()\mskip1.5mu\}{}\<[E]%
\ColumnHook
\end{hscode}\resethooks
%% 
with the interpretation $\interp{P; Q} = \interp{P} \ensuremath{\mathbin{`\Varid{mappend}`}} Q$ and
where \ensuremath{\Varid{mempty}} provides a \emph{no-op}. 

\subsection{Emitting overrun exceptions}

\paragraph{Overrun}

\paragraph{Overrun schedule}

\section{Related work}

There has been various work on reasoning about time in logic. For
example, modal CTL (Computational Tree Logic) has been extended with
time bounds for deadlines~\cite{emerson1991quantitative} (RCTL,
Real-Time Computational Tree Logic) and for soft deadlines with
probabilities on time bounds~\cite{hansson1994logic}. In these logics,
temporal modalities are indexed with time bounds, \eg{}, $AF^{\leq 50}
p$ means $p$ is true after at least $50$ ``time units'' (where $A$ is
the CTL connective for \emph{along all paths} and $F$ for
\emph{finally} (or \emph{eventually})). Our approach is less
prescriptive and explicit, but has some resemblance in the use of
\sleepOp{}. For example, the program $\sleep t ; P$ roughly
corresponds to $AF^{\leq t} \interp{P}$, \ie{}, after at leat $t$ then
whatever $P$ does will have happend. Our framework is not motivated by
logic and we do not have a model checking process for answering
questions such as, at time $t$ what formula hold (what statements have
been evluated).  The properties of Lemmas 1 to 5 however provide some
basis for programmers to reason about time in their programs. In
practice, we find that such reasoning can be done by children in a
completely informal but highly useful way.



\section{Epilogue}


\paragraph{Acknowledgements}

\bibliography{references}

\appendix

\paragraph{Proof} (of Theorem~\ref{theorem:main})
Since sequential composition is associative, we can reassociate
$P; Q$ such that $P = P_1; P'$ where $P_1$ is a single statement

\begin{itemize}
\item $P = \sleep t$

\begin{align*}
\begin{array}{ll}
       & \etime{\sleep t; Q} \\[0.5em]
\equiv & \; \{\textit{Lemma}~\ref{lem:sleep-L}\} \\[0.1em]
       & s + \etime{Q} \\[0.5em]
\equiv & \; \{\textit{Definition}~\ref{}\} \\[0.1em]
       & \etime{\sleep t} + \etime{Q}
\end{array}
\end{align*}

\item $Q = \sleep t$ by Lemma~\ref{lem:sleep-R} then 
there are two cases:

\begin{itemize}
\item $\etime{P} \leq t$ then:

\begin{align*}
\begin{array}{ll}
       & \etime{P; \sleep t} \\[0.5em]
\equiv & \; \{\textit{Lemma}~\ref{lem:sleep-R}\} \\[0.1em]
       & \vtime{P} + t \\[0.5em]
\equiv & \; \{\textit{Definition}~\ref{sleep-spec}\} \\[0.1em]
       & \vtime{P} + \vtime{\sleep t}
\end{array}
\end{align*}

\item $\etime{P} > t$ then:

\begin{align*}
\begin{array}{ll}
       & \etime{P; \sleep t} \\[0.5em]
\equiv & \; \{\textit{Lemma}~\ref{lem:sleep-R}\} \\[0.1em]
       & \etime{P} \\[0.5em]
\leq   & \etime{P} + \etime{\sleep t}
\end{array}
\end{align*}

\end{itemize}

\item $P = P';P'', Q = Q';Q''$

Ressociate so that $P'; (P''; (Q'; Q''))$ then

By induction:
%%
\begin{align}
\vtime{P'} + \vtime{P''} \leq \etime{P'; P''} \leq \vtime{P'} + \vtime{P''} \\
\vtime{Q'} + \vtime{Q''} \leq \etime{Q'; Q''} \leq \vtime{Q'} + \vtime{Q''} \\
\end{align}

and
\note{STUCK!}
\end{itemize}


\end{document}
