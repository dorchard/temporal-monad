\documentclass[preprint]{sigplanconf}

%% ODER: format ==         = "\mathrel{==}"
%% ODER: format /=         = "\neq "
%
%
\makeatletter
\@ifundefined{lhs2tex.lhs2tex.sty.read}%
  {\@namedef{lhs2tex.lhs2tex.sty.read}{}%
   \newcommand\SkipToFmtEnd{}%
   \newcommand\EndFmtInput{}%
   \long\def\SkipToFmtEnd#1\EndFmtInput{}%
  }\SkipToFmtEnd

\newcommand\ReadOnlyOnce[1]{\@ifundefined{#1}{\@namedef{#1}{}}\SkipToFmtEnd}
\usepackage{amstext}
\usepackage{amssymb}
\usepackage{stmaryrd}
\DeclareFontFamily{OT1}{cmtex}{}
\DeclareFontShape{OT1}{cmtex}{m}{n}
  {<5><6><7><8>cmtex8
   <9>cmtex9
   <10><10.95><12><14.4><17.28><20.74><24.88>cmtex10}{}
\DeclareFontShape{OT1}{cmtex}{m}{it}
  {<-> ssub * cmtt/m/it}{}
\newcommand{\texfamily}{\fontfamily{cmtex}\selectfont}
\DeclareFontShape{OT1}{cmtt}{bx}{n}
  {<5><6><7><8>cmtt8
   <9>cmbtt9
   <10><10.95><12><14.4><17.28><20.74><24.88>cmbtt10}{}
\DeclareFontShape{OT1}{cmtex}{bx}{n}
  {<-> ssub * cmtt/bx/n}{}
\newcommand{\tex}[1]{\text{\texfamily#1}}	% NEU

\newcommand{\Sp}{\hskip.33334em\relax}


\newcommand{\Conid}[1]{\mathit{#1}}
\newcommand{\Varid}[1]{\mathit{#1}}
\newcommand{\anonymous}{\kern0.06em \vbox{\hrule\@width.5em}}
\newcommand{\plus}{\mathbin{+\!\!\!+}}
\newcommand{\bind}{\mathbin{>\!\!\!>\mkern-6.7mu=}}
\newcommand{\rbind}{\mathbin{=\mkern-6.7mu<\!\!\!<}}% suggested by Neil Mitchell
\newcommand{\sequ}{\mathbin{>\!\!\!>}}
\renewcommand{\leq}{\leqslant}
\renewcommand{\geq}{\geqslant}
\usepackage{polytable}

%mathindent has to be defined
\@ifundefined{mathindent}%
  {\newdimen\mathindent\mathindent\leftmargini}%
  {}%

\def\resethooks{%
  \global\let\SaveRestoreHook\empty
  \global\let\ColumnHook\empty}
\newcommand*{\savecolumns}[1][default]%
  {\g@addto@macro\SaveRestoreHook{\savecolumns[#1]}}
\newcommand*{\restorecolumns}[1][default]%
  {\g@addto@macro\SaveRestoreHook{\restorecolumns[#1]}}
\newcommand*{\aligncolumn}[2]%
  {\g@addto@macro\ColumnHook{\column{#1}{#2}}}

\resethooks

\newcommand{\onelinecommentchars}{\quad-{}- }
\newcommand{\commentbeginchars}{\enskip\{-}
\newcommand{\commentendchars}{-\}\enskip}

\newcommand{\visiblecomments}{%
  \let\onelinecomment=\onelinecommentchars
  \let\commentbegin=\commentbeginchars
  \let\commentend=\commentendchars}

\newcommand{\invisiblecomments}{%
  \let\onelinecomment=\empty
  \let\commentbegin=\empty
  \let\commentend=\empty}

\visiblecomments

\newlength{\blanklineskip}
\setlength{\blanklineskip}{0.66084ex}

\newcommand{\hsindent}[1]{\quad}% default is fixed indentation
\let\hspre\empty
\let\hspost\empty
\newcommand{\NB}{\textbf{NB}}
\newcommand{\Todo}[1]{$\langle$\textbf{To do:}~#1$\rangle$}

\EndFmtInput
\makeatother
%
%
%
%
%
%
% This package provides two environments suitable to take the place
% of hscode, called "plainhscode" and "arrayhscode". 
%
% The plain environment surrounds each code block by vertical space,
% and it uses \abovedisplayskip and \belowdisplayskip to get spacing
% similar to formulas. Note that if these dimensions are changed,
% the spacing around displayed math formulas changes as well.
% All code is indented using \leftskip.
%
% Changed 19.08.2004 to reflect changes in colorcode. Should work with
% CodeGroup.sty.
%
\ReadOnlyOnce{polycode.fmt}%
\makeatletter

\newcommand{\hsnewpar}[1]%
  {{\parskip=0pt\parindent=0pt\par\vskip #1\noindent}}

% can be used, for instance, to redefine the code size, by setting the
% command to \small or something alike
\newcommand{\hscodestyle}{}

% The command \sethscode can be used to switch the code formatting
% behaviour by mapping the hscode environment in the subst directive
% to a new LaTeX environment.

\newcommand{\sethscode}[1]%
  {\expandafter\let\expandafter\hscode\csname #1\endcsname
   \expandafter\let\expandafter\endhscode\csname end#1\endcsname}

% "compatibility" mode restores the non-polycode.fmt layout.

\newenvironment{compathscode}%
  {\par\noindent
   \advance\leftskip\mathindent
   \hscodestyle
   \let\\=\@normalcr
   \(\pboxed}%
  {\endpboxed\)%
   \par\noindent
   \ignorespacesafterend}

\newcommand{\compaths}{\sethscode{compathscode}}

% "plain" mode is the proposed default.
% It should now work with \centering.
% This required some changes. The old version
% is still available for reference as oldplainhscode.

\newenvironment{plainhscode}%
  {\hsnewpar\abovedisplayskip
   \advance\leftskip\mathindent
   \hscodestyle
   \let\hspre\(\let\hspost\)%
   \pboxed}%
  {\endpboxed%
   \hsnewpar\belowdisplayskip
   \ignorespacesafterend}

\newenvironment{oldplainhscode}%
  {\hsnewpar\abovedisplayskip
   \advance\leftskip\mathindent
   \hscodestyle
   \let\\=\@normalcr
   \(\pboxed}%
  {\endpboxed\)%
   \hsnewpar\belowdisplayskip
   \ignorespacesafterend}

% Here, we make plainhscode the default environment.

\newcommand{\plainhs}{\sethscode{plainhscode}}
\newcommand{\oldplainhs}{\sethscode{oldplainhscode}}
\plainhs

% The arrayhscode is like plain, but makes use of polytable's
% parray environment which disallows page breaks in code blocks.

\newenvironment{arrayhscode}%
  {\hsnewpar\abovedisplayskip
   \advance\leftskip\mathindent
   \hscodestyle
   \let\\=\@normalcr
   \(\parray}%
  {\endparray\)%
   \hsnewpar\belowdisplayskip
   \ignorespacesafterend}

\newcommand{\arrayhs}{\sethscode{arrayhscode}}

% The mathhscode environment also makes use of polytable's parray 
% environment. It is supposed to be used only inside math mode 
% (I used it to typeset the type rules in my thesis).

\newenvironment{mathhscode}%
  {\parray}{\endparray}

\newcommand{\mathhs}{\sethscode{mathhscode}}

% texths is similar to mathhs, but works in text mode.

\newenvironment{texthscode}%
  {\(\parray}{\endparray\)}

\newcommand{\texths}{\sethscode{texthscode}}

% The framed environment places code in a framed box.

\def\codeframewidth{\arrayrulewidth}
\RequirePackage{calc}

\newenvironment{framedhscode}%
  {\parskip=\abovedisplayskip\par\noindent
   \hscodestyle
   \arrayrulewidth=\codeframewidth
   \tabular{@{}|p{\linewidth-2\arraycolsep-2\arrayrulewidth-2pt}|@{}}%
   \hline\framedhslinecorrect\\{-1.5ex}%
   \let\endoflinesave=\\
   \let\\=\@normalcr
   \(\pboxed}%
  {\endpboxed\)%
   \framedhslinecorrect\endoflinesave{.5ex}\hline
   \endtabular
   \parskip=\belowdisplayskip\par\noindent
   \ignorespacesafterend}

\newcommand{\framedhslinecorrect}[2]%
  {#1[#2]}

\newcommand{\framedhs}{\sethscode{framedhscode}}

% The inlinehscode environment is an experimental environment
% that can be used to typeset displayed code inline.

\newenvironment{inlinehscode}%
  {\(\def\column##1##2{}%
   \let\>\undefined\let\<\undefined\let\\\undefined
   \newcommand\>[1][]{}\newcommand\<[1][]{}\newcommand\\[1][]{}%
   \def\fromto##1##2##3{##3}%
   \def\nextline{}}{\) }%

\newcommand{\inlinehs}{\sethscode{inlinehscode}}

% The joincode environment is a separate environment that
% can be used to surround and thereby connect multiple code
% blocks.

\newenvironment{joincode}%
  {\let\orighscode=\hscode
   \let\origendhscode=\endhscode
   \def\endhscode{\def\hscode{\endgroup\def\@currenvir{hscode}\\}\begingroup}
   %\let\SaveRestoreHook=\empty
   %\let\ColumnHook=\empty
   %\let\resethooks=\empty
   \orighscode\def\hscode{\endgroup\def\@currenvir{hscode}}}%
  {\origendhscode
   \global\let\hscode=\orighscode
   \global\let\endhscode=\origendhscode}%

\makeatother
\EndFmtInput
%


\usepackage{fancyvrb}
\usepackage{stmaryrd}
\usepackage{amsthm}
\usepackage{amsmath}
\usepackage{xypic}
\usepackage{multirow}
\usepackage{hyperref}
\usepackage{url}
\usepackage{color}

\ifdefined\nolhs
\DefineVerbatimEnvironment{code}{Verbatim}{fontsize=\small}
\else
\fi

\bibliographystyle{amsalpha}

\newcommand{\note}[1]{{\color{blue}{#1}}}

\newtheorem{lemma}{Lemma}
\theoremstyle{definition}
\newtheorem{definition}{Definition}

\newcommand{\play}{\mathsf{play}\;}
\newcommand{\sleep}{\mathsf{sleep}\;}
\newcommand{\playOp}{\textsf{play}}
\newcommand{\sleepOp}{\textsf{sleep}}
\newcommand{\lang}{SonicPi}

\newcommand{\vtime}[1]{[#1]_{\mathsf{v}}}
\newcommand{\etime}[1]{[#1]_{\mathsf{t}}}

\newcommand{\interp}[1]{\llbracket{#1}\rrbracket}

\newcommand{\ie}{\emph{i.e.}}
\newcommand{\eg}{\emph{e.g.}}

\authorinfo{Sam Aaron}{}{}
\authorinfo{Dominic Orchard}{}{}
\title{A programming model for temporal coordination (in music)}

\begin{document}
\maketitle

\begin{figure}[t]
\[
\begin{array}{l}
\play C \\
\play E \\ 
\play G \\
\sleep 1 \\
\play F \\
\play A \\
\play C \\
\sleep 0.5 \\
\play G \\
\play B \\
\play D
\end{array}
\]
\caption{Playing three chords in \lang{} with the second two chords played
closer together by $0.5s$.}
\end{figure}

The \sleepOp{} statement essentially imposes a minimum \emph{soft deadline}
(in the terminology of )

Thus ``$\sleep{} t$'' communicates that, after it has been evaluated, at least 
$t$ seconds has elapsed since the last \sleepOp{}. This provides a kind of
\emph{minimum (soft) deadline}. 

In \lang{}, it is possible that a computation proceeding a \sleepOp{}
can overrun; that is, run longer than the sleep time.  Thus, the
programming model is not suitable for realtime systems requiring hard
deadlines but \sleepOp{} instead provides a \emph{soft deadline} (using
the terminology of Hansson and Jonsson~\cite{hansson1994logic}).



\paragraph{Terminology} 
We refer to closed (\ie{}, without free variables)
sequences of statements as \emph{programs}.

\begin{definition}
For a program $P$, then $\vtime{P}$ is the \emph{virtual time} that
elapses when running $P$ and $\etime{P}$ is the \emph{actual (kernel)
  time} that elapses when running $P$.
\end{definition}

\begin{definition}
\note{This is partial, there is no compositional definition for
$\etime{-}$ (proving this might be interesting, it can be
done by example).}
Since the \sleepOp{} operation increments virtual time, but also
causes kernel sleep then: $\vtime{\sleep t} = \etime{\sleep t} = t$,
and
%
\[
\vtime{P; Q} = \vtime{P} + \vtime{Q}
\]
\end{definition}

\begin{lemma}
For all programs $P$ then $\etime{P} \geq \vtime{P}$. 
\label{lemma1}
\end{lemma}

\begin{lemma}
For all programs $P$ and $Q$ then:
%%
\begin{equation}
\vtime{P} + \vtime{Q} \leq \etime{P; Q} \leq \etime{P} + \etime{Q}
\end{equation}
\label{lemma2}
\end{lemma}

\paragraph{Proof}
TODO

\begin{lemma}
For a program $P$ then:
%%
\begin{align*}
\etime{P; \sleep{} s} = 
 \begin{cases}
   s + \vtime{P} & \etime{P} \leq s \\
   \etime{P} & \etime{P} > s
 \end{cases}
\end{align*}
\label{lemma3}
\end{lemma}

\begin{lemma}
For a program $P$ then
%%
\begin{align*}
\etime{\sleep{} s; P} = s + \etime{P}
\end{align*}
\label{lemma4}
\end{lemma}


For example, consider subprograms $A$, $B$, $C$ interposed with two
sleep statements with lengths $s_1$ and $s_2$:
%
\begin{equation}
\begin{array}{l}
A \\
\sleep s_1 \\
B  \\
\sleep s_2 \\
C
\end{array}
\label{example:time1}
\end{equation}
%%
If $[A] = t_1$, $[B] = t_2$, $[C] = t_3$, then $[eq. \eqref{example:time1}] = 
s_1 + s_2 + t_3$, iff $t_1 \leq 1$ and $t_2 \leq 2$.

%\begin{equation*}
%\begin{array}{lllll}
%A & \multirow{2}{*}{\rule[1em]{0.6pt}{1.2em}} & \multirow{2}{*}{$t_1$} & 
%\multirow{4}{*}{\rule[1em]{0.6pt}{4em}} & \multirow{4}{*}{$t_1 + t_2 + 3$}
%\\
%\emph{sleep} \; 1 \qquad \\
%B &  \multirow{2}{*}{\rule[1em]{0.6pt}{1.2em}} & \multirow{2}{*}{$t_2$} \\
%\emph{sleep} \; 2
%\end{array}
%\end{equation*}

\section{Model}

In the following, we use Haskell as our meta language for the
semantics (since it provides convenient syntax for working with
monads)\footnote{The source code for the model is avilable at
  \url{https://github.com/dorchard/time-monad}}.

\newcommand{\TM}{\mathsf{TM}}

\begin{hscode}\SaveRestoreHook
\column{B}{@{}>{\hspre}l<{\hspost}@{}}%
\column{5}{@{}>{\hspre}l<{\hspost}@{}}%
\column{E}{@{}>{\hspre}l<{\hspost}@{}}%
\>[B]{}\mathbf{data}\;\Conid{Temporal}\;\Varid{a}\mathrel{=}{}\<[E]%
\\
\>[B]{}\hsindent{5}{}\<[5]%
\>[5]{}\Conid{T}\;((\Conid{Time},\Conid{Time})\to (\Conid{VTime}\to \Conid{IO}\;(\Varid{a},\Conid{VTime}))){}\<[E]%
\ColumnHook
\end{hscode}\resethooks
%%
Which maps a pair of two times, which will be the
start time of the computation and current time, to
a stateful computation over the \emph{IO} monad with
a single location storing a virtual time. The \emph{IO}
computation provides underlying access to the actual
time from kernel. 


To model \lang{}, it is enough for \emph{Temporal} to have the
structure of an \emph{applicative functor}~\cite{mcbride2008functional} (also called
\emph{idioms} or \emph{monoidal functors}).  Recall the usual
applicative functor interface in Haskell:
%%
\begin{hscode}\SaveRestoreHook
\column{B}{@{}>{\hspre}l<{\hspost}@{}}%
\column{4}{@{}>{\hspre}l<{\hspost}@{}}%
\column{10}{@{}>{\hspre}l<{\hspost}@{}}%
\column{E}{@{}>{\hspre}l<{\hspost}@{}}%
\>[B]{}\mathbf{class}\;\Conid{Functor}\;\Varid{f}\Rightarrow \Conid{Applicative}\;\Varid{f}\;\mathbf{where}{}\<[E]%
\\
\>[B]{}\hsindent{4}{}\<[4]%
\>[4]{}\Varid{pure}{}\<[10]%
\>[10]{}\mathbin{::}\Varid{a}\to \Varid{f}\;\Varid{a}{}\<[E]%
\\
\>[B]{}\hsindent{4}{}\<[4]%
\>[4]{}(<\!\!\!*\!\!\!>)\mathbin{::}\Varid{f}\;(\Varid{a}\to \Varid{b})\to \Varid{f}\;\Varid{a}\to \Varid{f}\;\Varid{b}{}\<[E]%
\ColumnHook
\end{hscode}\resethooks
%%
The \emph{Applicative} interface describes how to compose effectul
computations encoded as values of type $f\ a$ (the effectful computation
of a pure value of type $a$). Thus, \emph{pure} constructs a trivially
effectful computation from a pure value. The \ensuremath{<\!\!\!*\!\!\!>} operation (which we'll
call \emph{apply}) takes an effectful computation of a function and
an effectful computation of an argument and evaluates the two effects


The \emph{Applicative} instance for \emph{Temporal} is:
%
\begin{hscode}\SaveRestoreHook
\column{B}{@{}>{\hspre}l<{\hspost}@{}}%
\column{5}{@{}>{\hspre}l<{\hspost}@{}}%
\column{21}{@{}>{\hspre}l<{\hspost}@{}}%
\column{30}{@{}>{\hspre}l<{\hspost}@{}}%
\column{33}{@{}>{\hspre}l<{\hspost}@{}}%
\column{44}{@{}>{\hspre}l<{\hspost}@{}}%
\column{E}{@{}>{\hspre}l<{\hspost}@{}}%
\>[B]{}\mathbf{instance}\;\Conid{Applicative}\;\Conid{Temporal}\;\mathbf{where}{}\<[E]%
\\
\>[B]{}\hsindent{5}{}\<[5]%
\>[5]{}\Varid{pure}\;\Varid{a}{}\<[21]%
\>[21]{}\mathrel{=}\Conid{T}\;(\lambda (\anonymous ,\anonymous )\to \lambda \Varid{vt}\to \Varid{return}\;(\Varid{a},\Varid{vt})){}\<[E]%
\\
\>[B]{}\hsindent{5}{}\<[5]%
\>[5]{}(\Conid{T}\;\Varid{f})<\!\!\!*\!\!\!>(\Conid{T}\;\Varid{x})\mathrel{=}\Conid{T}\;(\lambda (\Varid{startT},\Varid{nowT})\to \lambda \Varid{vT}\to {}\<[E]%
\\
\>[5]{}\hsindent{25}{}\<[30]%
\>[30]{}\mathbf{do}\;\Varid{thenT}{}\<[44]%
\>[44]{}\leftarrow \Varid{getCurrentTime}{}\<[E]%
\\
\>[30]{}\hsindent{3}{}\<[33]%
\>[33]{}(\Varid{f'},\Varid{vT'}){}\<[44]%
\>[44]{}\leftarrow \Varid{f}\;(\Varid{startT},\Varid{thenT})\;\Varid{vT}{}\<[E]%
\\
\>[30]{}\hsindent{3}{}\<[33]%
\>[33]{}(\Varid{x'},\Varid{vT''})\leftarrow \Varid{x}\;(\Varid{startT},\Varid{nowT})\;\Varid{vT'}{}\<[E]%
\\
\>[30]{}\hsindent{3}{}\<[33]%
\>[33]{}\Varid{return}\;(\Varid{f'}\;\Varid{x'},\Varid{vT''})){}\<[E]%
\ColumnHook
\end{hscode}\resethooks
%
The result of composing two temporal computations, with
start time \emph{starT}, current time \emph{nowT}, and virtual
time \emph{vT}, is the result of evaluating first the right 
hand side, and then the left hand side 

Note this gives a left-to-right evaluation order on the appicative
\emph{apply} operation. 

To understand this ordering, consider 


\begin{align*}
\interp{\emph{statement}} & : \emph{Temporal} \, () \\
\interp{P; Q} & = (\lambda() \rightarrow \interp{P}) <\!\!\!*\!\!\!> \interp{Q} \\
\interp{\sleep t} & = \emph{sleep} \, \interp{t}
\end{align*}
%%
Note that $\interp{-}$ is overloaded in the rule for \sleepOp{} for (pure) expressions. 
The concrete interpreation of other statements in the language, such as \playOp, is
elided here since it does not relate directly to the temporal semantics. 

\begin{figure}[t]
\begin{hscode}\SaveRestoreHook
\column{B}{@{}>{\hspre}l<{\hspost}@{}}%
\column{7}{@{}>{\hspre}l<{\hspost}@{}}%
\column{E}{@{}>{\hspre}l<{\hspost}@{}}%
\>[B]{}\Varid{time},\Varid{start}\mathbin{::}\Conid{Temporal}\;\Conid{Time}{}\<[E]%
\\
\>[B]{}\Varid{time}{}\<[7]%
\>[7]{}\mathrel{=}\Conid{T}\;(\lambda (\anonymous ,\Varid{nowT})\to \lambda \Varid{vT}\to \Varid{return}\;(\Varid{nowT},\Varid{vT})){}\<[E]%
\\
\>[B]{}\Varid{start}\mathrel{=}\Conid{T}\;(\lambda (\Varid{startT},\anonymous )\to \lambda \Varid{vT}\to \Varid{return}\;(\Varid{startT},\Varid{vT})){}\<[E]%
\\[\blanklineskip]%
\>[B]{}\Varid{getVirtualTime}\mathbin{::}\Conid{Temporal}\;\Conid{VTime}{}\<[E]%
\\
\>[B]{}\Varid{getVirtualTime}\mathrel{=}\Conid{T}\;(\lambda (\anonymous ,\anonymous )\to \lambda \Varid{vT}\to \Varid{return}\;(\Varid{vT},\Varid{vT})){}\<[E]%
\\[\blanklineskip]%
\>[B]{}\Varid{setVirtualTime}\mathbin{::}\Conid{VTime}\to \Conid{Temporal}\;(){}\<[E]%
\\
\>[B]{}\Varid{setVirtualTime}\;\Varid{vT}\mathrel{=}\Conid{T}\;(\lambda \anonymous \to \lambda \anonymous \to \Varid{return}\;((),\Varid{vT})){}\<[E]%
\ColumnHook
\end{hscode}\resethooks
\caption{Simple \emph{Temporal} computations, used  by the model}
\end{figure}

The key primitive \emph{sleep} provides the semantics for \sleepOp{} as:
%%
\begin{hscode}\SaveRestoreHook
\column{B}{@{}>{\hspre}l<{\hspost}@{}}%
\column{20}{@{}>{\hspre}l<{\hspost}@{}}%
\column{22}{@{}>{\hspre}l<{\hspost}@{}}%
\column{28}{@{}>{\hspre}l<{\hspost}@{}}%
\column{30}{@{}>{\hspre}l<{\hspost}@{}}%
\column{E}{@{}>{\hspre}l<{\hspost}@{}}%
\>[B]{}\Varid{sleep}\mathbin{::}\Conid{VTime}\to \Conid{Temporal}\;(){}\<[E]%
\\
\>[B]{}\Varid{sleep}\;\Varid{delayT}\mathrel{=}\mathbf{do}\;{}\<[20]%
\>[20]{}\Varid{vT}{}\<[28]%
\>[28]{}\leftarrow \Varid{getVirtualTime}{}\<[E]%
\\
\>[20]{}\Varid{startT}{}\<[28]%
\>[28]{}\leftarrow \Varid{start}{}\<[E]%
\\
\>[20]{}\Varid{nowT}{}\<[28]%
\>[28]{}\leftarrow \Varid{time}{}\<[E]%
\\
\>[20]{}\mathbf{let}\;\Varid{vT'}{}\<[30]%
\>[30]{}\mathrel{=}\Varid{vT}\mathbin{+}\Varid{delayT}{}\<[E]%
\\
\>[20]{}\mathbf{let}\;\Varid{diffT}\mathrel{=}\Varid{diffTime}\;\Varid{nowT}\;\Varid{startT}{}\<[E]%
\\
\>[20]{}\mathbf{if}\;(\Varid{vT'}\mathbin{<}\Varid{diffT}){}\<[E]%
\\
\>[20]{}\hsindent{2}{}\<[22]%
\>[22]{}\mathbf{then}\;\Varid{return}\;(){}\<[E]%
\\
\>[20]{}\hsindent{2}{}\<[22]%
\>[22]{}\mathbf{else}\;\Varid{kernelSleep}\;(\Varid{vT'}\mathbin{-}\Varid{diffT}){}\<[E]%
\\
\>[20]{}\Varid{setVirtualTime}\;\Varid{vT'}{}\<[E]%
\ColumnHook
\end{hscode}\resethooks
%%
\emph{sleep} proceeds by calculating the elapsed time, \emph{diffT}, 
and the new virtual time \emph{vT'}. If the new virtual time is less
than the elapsed time then no actual (kernel) sleeping happens. However,
if the new virtual time is ahead of the elapsed time, then the process
sleeps for the difference. Finally, the virtual time state is updated
with the new virtual time. 


\paragraph{Quotienting by non-time dependent functions}

In L, there is no expression which returns the current time; 
 \emph{getTime} belongs only to the model, not to the language.
That is, for all expressions $e$, then the denotation 
$\interp{e}$ factors through 


\subsection{Emitting overrun exceptions}

\paragraph{Overrun}

\paragraph{Overrun schedule}

\section{Related work}

There has been various work on reasoning about time in logic. For
example, modal CTL (Computational Tree Logic) has been extended with
time bounds for deadlines~\cite{emerson1991quantitative} (RCTL,
Real-Time Computational Tree Logic) and for soft deadlines with
probabilities on time bounds~\cite{hansson1994logic}. In these logics,
temporal modalities are indexed with time bounds, \eg{}, $AF^{\leq 50}
p$ means $p$ is true after at least $50$ ``time units'' (where $A$ is
the CTL connective for \emph{along all paths} and $F$ for
\emph{finally} (or \emph{eventually})). Our approach is less
prescriptive and explicit, but has some resemblance in the use of
\sleepOp{}. For example, the program $\sleep t ; P$ roughly
corresponds to $AF^{\leq t} \interp{P}$, \ie{}, after at leat $t$ then
whatever $P$ does will have happend. Our framework is not motivated by
logic and we do not have a model checking process for answering
questions such as, at time $t$ what formula hold (what statements have
been evluated).  The properties of Lemmas 1 to 5 however provide some
basis for programmers to reason about time in their programs. In
practise, we find that such reasoning can be done by children in a
completely informal but highly useful way.



\section{Epilogue}


\paragraph{Acknowledgements}

\bibliography{references}


\end{document}
